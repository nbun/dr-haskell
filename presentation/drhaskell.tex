\documentclass{beamer}
\usepackage{ulem}
\usepackage{amsmath}
\usepackage{amssymb}
\usepackage{subcaption}
\usepackage{graphicx}
\usepackage{listings}
\lstset{
	frame=none,
	xleftmargin=2pt,
	stepnumber=1,
	numbers=left,
	numbersep=5pt,
	numberstyle=\ttfamily\tiny\color[gray]{0.3},
	belowcaptionskip=\bigskipamount,
	captionpos=b,
	escapeinside={*'}{'*},
	language=haskell,
	tabsize=2,
	emphstyle={\bf},
	commentstyle=\it,
	stringstyle=\mdseries\rmfamily,
	showspaces=false,
	keywordstyle=\bfseries\rmfamily,
	columns=flexible,
	basicstyle=\small\sffamily,
	showstringspaces=false,
	morecomment=[l]\%,
}
\mode<presentation>
{
  \usetheme{metropolis}
}


\usepackage[german]{babel}
% or whatever

\usepackage[utf8]{inputenc}
% or whatever

\setbeamersize{text margin left=1.5em,text margin right=1.5em}

\title{Masterprojekt SS17 -- Dr. Haskell}


\author[shortname]{%\centering
	Niels Bunkenburg \and Jonas Busse \and Janina Harms \\
	\and Jan-Hendrik Matthes \and Marc André Wittorf}

\institute{ 
	Arbeitsgruppe für Programmiersprachen und Übersetzerkonstruktion \par
	Institut für Informatik \par
	Christian-Albrechts-Universität zu Kiel}

\date[Short Occasion]{\vfill\centering\today}

\begin{document}
\begin{frame}
	\titlepage
\end{frame}
\begin{frame}
	\frametitle{Das Projekt}
	\begin{itemize}
		\item Lernprogramm für Haskell im Kontext der Vorlesung 'FortProg' \pause
		\item Reduktion des Haskell-Sprachumfangs
		\item Verschiedene Sprachlevel
		\item Vereinfachung von Typen und Fehlermeldungen
	\end{itemize}
	Inspirationen \pause
	\begin{itemize}
		\item DrRacket
		\item Helium
	\end{itemize}
\end{frame}
\begin{frame}{Sprachlevel}
	Level beschränken den Umfang der Sprache, ähnlich wie bei DrRacket.
	\begin{itemize}
		\item Zusätzliche Einschränkungen  möglich, z.B. Angabe von Signaturen
		\item Implementierung als Teil der statischen Analyse
		\item Integration als Pragma: \texttt{\{-\# DRHASKELL LEVEL1 \#-\}}
	\end{itemize}
	Vorhandene Level: \par
	\begin{minipage}[T]{.55 \textwidth}
		\begin{itemize}
			\item 1: Die Basics
			\item 2: Datentypen, Polymorphie und Higher-Order
			\item 3: Eigene Module
			\item 4: Voller Sprachumfang
		\end{itemize}
	\end{minipage}
	\begin{minipage}[T]{.4 \textwidth}
			\scalebox{6}{\}} Eigene Typinferenz \par \hspace{32px} ohne Typklassen
	\end{minipage}
\end{frame}

\begin{frame}[fragile]
	\begin{lstlisting}
		module Test where
		test = 42
	\end{lstlisting}
\end{frame}
\begin{frame}
	\frametitle{Softwarearchitektur}
\end{frame}
\begin{frame}[fragile]
\frametitle{DrHaskell -- Beispiel}
\lstinputlisting{../examples/Tree.hs}
\end{frame}

\begin{frame}
	\frametitle{DrHaskell -- Konvertierung HSE nach AH}
	\begin{itemize}
		\item Transformation von HSE zu AH
		  \begin{itemize}
		    \item Umgang mit Funktionen
		    \item Umgang mit Variablen
		  \end{itemize}
		\item Ergänzen von Argumenten
		\item Abstrakte Repräsentation bilden
		\item Liften der Funktionen
	\end{itemize}
\end{frame}

\begin{frame}[fragile]
\frametitle{DrHaskell -- Konvertiertes Beispiel}
\lstinputlisting{../examples/TreeConv.hs}
\end{frame}

\begin{frame}
  \frametitle{DrHaskell -- Typinferenz}
  \begin{itemize}
    \item Grundlagen
    \item Preludefunktionen und Datentypen
    \item Nicht getypte Funktionen
    \item Typsignaturen in Typumgebung aufnehmen
    \item Restliches Programm (mit Signaturen)
    \item \glqq{}Too General Check\grqq{}
  \end{itemize}
\end{frame}

\begin{frame}
	\frametitle{Tests und Intermediate Format}
	\begin{itemize}
		\item Manuelles Testen?
		\item Test Suites?
		\vspace{1em}\pause
		\item Check Expect wie in DrRacket!
		\item Integration von QuickCheck!
		\vspace{1em}\pause
		\item Zusätzliche Konstrukte
		\item Umformung in ein Zwischenformat
	\end{itemize}
	\vspace{1em}\pause
	$\Rightarrow$ Beispiel
\end{frame}

\begin{frame}
	\frametitle{Code Coverage}
	\begin{itemize}
		\item Was führen wir aus? \hspace{2em} \only<1->{Unsere Testfälle}
		\item Womit? \hspace{7.2em} \only<2->{HPC}
		\item Wie zeigen wir an? \hspace{2.65em} \only<3->{Linter}
	\end{itemize}
\end{frame}

\begin{frame}
	\frametitle{Code Coverage II}
	Ablauf:
	\begin{enumerate}
		\item Umformung in das Zwischenformat \pause
		\item Kompilieren mit \texttt{-fhpc} \pause
		\item Ausführen des Kompilats \pause
		\item Umrechnen der HPC-spezifischen \texttt{*.mix}- und \texttt{*.tix}-Dateien in ausgeführte Codestellen \pause
		\item Ausgabe mit anderen Linter-Meldungen
	\end{enumerate}
\end{frame}
\begin{frame}
	\frametitle{Installation}
	\begin{itemize}
		\item{Zwei Möglichkeiten} \pause
		\item{Cabal}
			\begin{enumerate}
				\item{Klonen des Git-Repositories}
				\item{cabal sandbox init}
				\item{cabal install}
				\item{Direktes Aufrufen der Programme aus der Cabal-Sandbox}
			\end{enumerate}
			\pause
		\item{Docker}
			\begin{enumerate}
				\item{docker pull jonasbusse/drhaskell}
				\item{Benutzung über vorgefertigtes DockerCall-Skript}
			\end{enumerate}
	\end{itemize}
\end{frame}
\begin{frame}
	\frametitle{Fazit}
	\begin{itemize}
		\item{Erweiterbare REPL mit eigener Typinferenz und Testrunner}
		\item{Erweiterbares Stufensystem}
		\item{Ausführen statischer Checks vor dem Laden}
		\item{Leichter verständliche Fehlermeldungen durch Einschränkung der Möglichkeiten}
		\item{Einfaches Schreiben von Testfällen}
		\item{Linter mit CodeCoverage-Anzeige der Testfälle}
		\item{Unterstützt den Benutzer beim Lernen von Haskell}
	\end{itemize}
\end{frame}
\begin{frame}
	\frametitle{Ausblick}
	\begin{itemize}
		\item{Typinferenz mit Typklassen}
		\item{Vorbedingungen für CheckExpect-Syntax}
		\item{Benutzerdefinierte Levelkonfiguration}
	\end{itemize}
\end{frame}
\end{document}


