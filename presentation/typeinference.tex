\begin{frame}
	\frametitle{DrHaskell -- Die Konvertierung}
	\begin{itemize}
		\item Transformation von HSE zu AH
		  \begin{itemize}
		    \item Umgang mit Funktionen
		    \item Umgang mit Variablen
		  \end{itemize}
		\item Ergänzen von Argumenten
		\item Abstrakte Repräsentation bilden
		\item Liften der Funktionen
	\end{itemize}
\end{frame}

\begin{frame}[fragile]
\frametitle{DrHaskell -- Die Beispiele}
  \begin{lstlisting}
  data Tree a = Leaf a | Branch (Tree a) (Tree a)

  mirrorTree l@(Leaf _)   = l
  mirrorTree (Branch l r) = Branch (mirrorTree r) (mirrorTree l)
  
  sumTree2 :: Tree Int -> Int
  sumTree2 = sumTree 0
    where  sumTree k (Leaf n) = n + k
           sumTree k (Branch l r) = sumTree (sumTree k l) r 

  \end{lstlisting}
\end{frame}

\begin{frame}
  \frametitle{DrHaskell -- Die Typeinferenz}
  \begin{itemize}
    \item Grundlagen
    \item Preludefunktionen und Datetypen
    \item Nicht getypte Funktionen
    \item Typsignaturen in Typeumgebung aufnehmen
    \item Restliches Programm (mit Signaturen)
    \item "Too General Check"
  \end{itemize}
\end{frame}
