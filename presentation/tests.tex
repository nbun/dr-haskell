\begin{frame}
	\frametitle{Tests und transformierter Quellcode}
	\begin{itemize}
		\item Manuelles Testen?
		\item Test Suites?
		\vspace{1em}\pause
		\item \texttt{check-expect} wie in DrRacket!
		\item Integration von QuickCheck!
		\vspace{1em}\pause
		\item Zusätzliche Konstrukte
		\item Anreicherung des Quelltextes um Testmechanismus
	\end{itemize}
\end{frame}
\begin{frame}
	\frametitle{QuickCheck Arbitrary?}
	\only<1>{QuickCheck braucht}
	%Here be dragons...
	\only<2->{\parbox{\widthof{QuickCheck braucht}-\widthof{Wir generieren}}{\hspace{-100cm}a} Wir generieren}
	Arbitrary-Instanzen, um eigene Datentypen testen zu können!\only<1>{?}
	
	\begin{itemize}
		\item Naiv
		\item Betrachtet alle Konstruktoren
		\item Erzeugt für jeden einen \texttt{Gen}-Ausdruck
		\item $\Rightarrow$ ruft \texttt{arbitrary} für alle Argumente rekursiv auf
		\item Verknüpft sie per \texttt{oneof}
	\end{itemize}
	
	$\Rightarrow$ Beispiel
\end{frame}
\begin{frame}
	\frametitle{Code Coverage}
	\begin{itemize}
		\item Was führen wir aus? \hspace{2em} \only<1->{Unsere Testfälle}
		\item Womit? \hspace{7.2em} \only<2->{HPC}
		\item Wie zeigen wir an? \hspace{2.65em} \only<3->{Linter}
	\end{itemize}
\end{frame}

\begin{frame}
	\frametitle{Code Coverage II}
	Ablauf:
	\begin{enumerate}
		\item Umformung in das Zwischenformat
		\item Kompilieren mit \texttt{-fhpc}
		\item Ausführen des Kompilats
		\item Umrechnen der HPC-spezifischen \texttt{*.mix}- und \texttt{*.tix}-Dateien in ausgeführte Codestellen
		\item Ausgabe mit anderen Linter-Meldungen
	\end{enumerate}
\end{frame}